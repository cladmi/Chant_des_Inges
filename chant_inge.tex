\documentclass[10pt]{book}
\usepackage[lyric]{songs}
\usepackage{graphicx}
\usepackage[latin1]{inputenc}  %% les accents dans le fichier.tex

\usepackage[a6paper]{geometry}


\nosongnumbers		% disables song numbering (in the grey box)
\pagestyle{empty}	% no line numbers
%\setlength{\sbarheight}{0pt}	% no funcking bar over and under each song

\versesep=14.5pt
\songcolumns{1}


\textheight=13cm
\voffset=-0.75cm
\hoffset=-0.5cm
\textwidth=8.75cm

\newindex{titleidx}{titleidx}
\noversenumbers

\renewcommand{\notefont}{\rmfont\Large}


\begin{document}
\begin{songs}{titleidx}
\beginsong{Chant des Ing�nieurs Grenoblois}[by={\small Par Passoire, compl�t� par les Ing�s Grenoblois 2011}, sr={Sur l'air de "When Johnny comes marching home"}, cr={~\LaTeX~version by Alt~F4}]

{\small

		\beginverse
		C'est nous les ing�s grenoblois, Hourra\,! Hourra\,!
		Quand on nous voit, on dit~: "Ces mecs\,!", Hourra\,! Hourra\,!
		Sont des guindailleurs, sont des s�ducteurs,
		Les plus grands buveurs, toujours mijoleurs,
		Ing�nieurs, oui, peut-�tre, un jour nous serons.
		\endverse

		\beginverse
		Parmi nous y a les Phelma Hourra\,! Hourra\,!
		Leur physique vous interpellera, Hourra\,! Hourra\,!
		Nano-�lectronique, mais pas nano-foies
		Jusqu'au bout de la nuit et � chaque fois
		Ing�nieurs, oui, peut-�tre, un jour nous serons.
		\endverse

		\beginverse
		Ensuite viennent les Ense3, Hourra\,! Hourra\,!
		Au grand jamais ils ne titubent, Hourra\,! Hourra\,!
		Ils induisent en vous un flux �lectrique
		Mesdames, venez pomper leur fluid'magique
		Ing�nieurs, oui, peut-�tre, un jour nous serons.
		\endverse

		\beginverse
		Mais non Papet', ils sont pas morts, Hourra\,!, Hourra\,!
		�a se se saurait si c'�tait le cas, Hourra\,! Hourra\,!
		Que ce soit au taff, que ce soit au bar
		La papeterie, proche de l'orgie
		Ing�nieurs, oui, peut-�tre, un jour nous serons
		\endverse

		\beginverse
		Loin � Valence les Esisar, Hourra\,! Hourra\,!
		N'en sont pas les derniers pour boire, Hourra\,! Hourra\,!
		Embarquez pour le syst�me 7�me ciel
		C'est bien avec eux que la vie est belle
		Ing�nieurs, oui, peut-�tre, un jour nous serons.
		\endverse

		\newpage

		\beginverse
		Au centre ville, y'a les GI, Hourra\,!, Hourra\,!
		Eux leur dada, c'est l'industrie, Hourra\,!, Hourra\,!
		Mais pour picoler, jamais les derniers,
		Quand ils sont l�, on ne s'ennuie pas,
		Ing�nieur, oui, peut-�tre, un jour nous serons.
		\endverse

		\beginverse
		Et depuis peu, y'a Polytech', Hourra\,!, Hourra\,!
		Toujours op�' pour faire des secs, Hourra\,!, Hourra\,!
		Pr�vention des Risques, Elec ou R�seaux
		Chez eux il y a tout ce qu'il vous faut
		Ing�nieur, oui, peut-�tre un jour nous serons.
		\endverse

		\beginverse
		Finalement il y a les Imag, Hourra\,! Hourra\,!
		Vous les croyez nerd' la bonne blague, Hourra\,! Hourra\,!
		Travaillant pour eux les ordinateurs
		Nous permettent � tous de chanter en choeur,
		Ing�nieurs, oui, peut-�tre, un jour nous serons.
		\endverse

		\beginverse
		Mais nous restons tous tr�s unis, Hourra\,! Hourra\,!
		Des derni�res aux premi�res ann�es, Hourra\,! Hourra\,!
		C'est nous les ing�s grenoblois.
		Et jusqu'� la mort, nous boirons encore,
		Ing�nieurs, oui, toujours, nous le resterons.
		\endverse
}

\endsong
\end{songs}

\end{document}
